\section{Análisis del problema} % (fold)
\label{sec:an_lisis_del_problema}

A la hora de abordar el problema debemos definir claramente los roles que va
a tener cada dispositivo en nuestro sistema. Un forma podría ser la siguiente.

\begin{itemize}
	\item La Raspberry Pi es el componente idóneo para realizar las tareas de
	obtención de datos, puesto que en ella corre un sistema operativo que nos
	facilita mucho las tareas de las conexiones a Internet. Si quisiésemos usar
	un arduino para esto sería mucho más complejo.
	\item Usaremos un Arduino para la parte del manejo del LCD. Es decir, el
	arduino simplemente se limitará a recibir una cadena de texto y mostrarla
	por la pantalla.
\end{itemize}

Este escenario descrito es muy común. Habitualmente se usará una
\texttt{Raspberry Pi} como el elemento que procesa y obtiene datos y dejaremos
para un \texttt{Arduino} las tareas relacionadas con el control de dispositivos,
en este caso un LCD.

\subsection{Obtención de datos} % (fold)
\label{sub:obtenci_n_de_datos}

Empecemos por la parte que deberá obtener los datos (tuits) de Internet.

En primer lugar debemos pensar en cómo se pueden obtener los datos del servidor
de Twitter. Podríamos hacer una aplicación en Python que obtuviese tuits del
servidor del Twitter. Esta parte se da ya hecha en este ejemplo, puesto que queda
fuera del ámbito de la electrónica y es algo más relacionado con las
comunicaciones en Internet.

Por lo tanto, dicho esto, vamos a partir de una aplicación en Python que ya
consigue obtener los tuits y mostrarlos por consola. Modificaremos este
comportamiento para que, en lugar de mostrarlos simplemente por consola, también
los envíe a nuestro \texttt{Arduino} para que este pueda mostrarlo en el panel
LCD.

\subsection{El protocolo MQTT} % (fold)
\label{sub:el_protocolo_mqtt}

Ya que suponemos que tenemos nuestro tuit, ¿cómo podemos enviarlo a un Arduino?
Podríamos simplemente enviar la cadena de texto a través de una conexión y
funcionaría perfectamente. Sin embargo, imaginemos la siguiente situación:

Queremos tener varios \texttt{Arduinos} (\texttt{Arduino} A y \texttt{Arduino B}) y varias
\texttt{Raspberries Pi} (A, B y C) en nuestro sistema.

Cada una de las \texttt{Raspberry Pi} obtiene los tuits de una cuenta diferente
y queremos, por ejemplo, que en \texttt{Arduino} A se muestre los tuits de las
\texttt{Raspberry} A y B y en \texttt{Arduino} B queremos que se muestren los
tuits de todas las \texttt{Raspberries}.

La cosa se vuelve mucho más complicada. Cuando ahora una \texttt{Raspberry} recibe
un tuit, ¿a quién se le envía?. Para poder realizarlo de esta forma, cada dispositivo
debe conocer la configuración del dispositivo remoto. Si el sistema estuviese
compuesto por muchos dispostivos sería casi imposible.

Para eso existen los \textbf{protocolos de nivel de aplicación}. Estos protocolos
funcionan sobre los dispositivos que usamos para enviar datos, como Bluetooth y
WiFi y nos permiten simplificar las comunicaciones. En nuestro caso usaremos
\textbf{MQTT} como protocolo de nivel de aplicación por ser muy sencillo y fácil
de implementar en \texttt{Arduino} y en \texttt{Raspberry Pi}.

MQTT nos permite crear un \texttt{broker} donde se
conecten múltiples clientes. Existen dos tipos de clientes, los publicadores y
los suscriptores. Además en cada broker la información se organiza en diferenes
\texttt{topics}.

El funcionamiento es muy sencillo, un cliente se conecta a un \texttt{broker} y
puede \textbf{publicar} un mensaje en un determinado \texttt{topic}. Un cliente
también puedeo \textbf{suscribirse} a un \texttt{topic}.

De esta forma, todos los clientes que estén suscrito a un \texttt{topic} recibiŕan
todos los mensajes que otros clientes publiquen en dicho \texttt{topic}. Esto se
conoce como un protocolo \textbf{PUBLISH/SUSCRIBE}.

Si volvemos a nuestro escenario, parece que encaja perfectamente todo. Podríamos
instalar un \texttt{broker} en nuestra \texttt{Raspberry Pi} al que se conecten
los \texttt{Arduinos} como \textbf{suscriptores} y nuestra aplicación en
\texttt{Python} como \textbf{publicadores}. Cuando obtenemos un tuit del
servidor de Twitter de la cuenta ``Usuario1'' podemos enviarlo al \texttt{topic}
``Usuario1''. De esta forma, todos los \texttt{Arduinos} que estén suscritos
al \texttt{topic} ``Usuario1'' recibirán inmediátamente el tuit.

\subsection{Bluetooth} % (fold)
\label{sub:bluetooth}

Ya hemos definido nuestro protocolo de nivel de aplicación, pero ¿cómo viaje el
mensaje que hemos creado desde nuestra \texttt{Raspberry Pi} hasta nuestro
\texttt{Arduino}?

Necesitaremos usar algún dispositivo que nos permita la comunicación. Para ello
vamos a usar \textbf{Módulos Bluetooth}. Estos módulo utilizan el protocolo
Bluetooth para ofrecernos una comunicación bidirecional.

El funcionamiento es muy sencillo ya que es exactamente igual que enviar datos
por puerto serie, de hecho, en nuestro \texttt{Arduino} el módulo estará
directamente conectado al puerto serie, por lo que no notaremos ninguna
diferencia. En cuanto a la \texttt{Raspberry Pi} no necesitaremos comunicarnos
directamente con el módulo Bluetooth ya que para ello se encargará un componente
de MQTT.

Recordemos que para enviar un dato por puerto serie usábamos la función:

\texttt{DigitalWrite("Hola Mundo")}.

\subsection{Panel LCD} % (fold)
\label{sub:panel_lcd}

Usaremos un panel T6963C, que tiene una resolución de  160x128 pixeles monocromático
que conectaremos a nuestro \texttt{Arduino}. Para comunicarnos con él usaremos
también el puerto serie.

Puesto que  el \texttt{Arduino UNO} sólamente dispone
de un único puerto serie, tendremos que usar una emulación de puerto serie
en otros pines para poder conectar el panel. Usar un puerto serie emulado
tiene algunas desventajas, pero en este caso no será importante. El funcionamiento
de un puerto serie emulado es idéntico al de un puerto serie estándar.

En esta pantalla imprimiremos el último tuit que hayamos recibido desde la
\texttt{Raspberry Pi} con el siguiente formato:

\begin{verbatim}
@ESIBot 18:57 el 5/10/2015
Hoy han comenzado las actividades para este curso con el
1er seminario de conceptos básicos. Mañana repetimos, de
16 a 18 en la 110 ;)
\end{verbatim}